% This is LLNCS.DEM the demonstration file of
% the LaTeX macro package from Springer-Verlag
% for Lecture Notes in Computer Science,
% version 2.4 for LaTeX2e as of 16. April 2010
%
\documentclass{llncs}
%
\usepackage{makeidx}  % allows for indexgeneration
%
\begin{document}

\title{Automatic Estimation of the Optic Nerve Sheat Diameter from Ultrasound Images}
%
\titlerunning{Automatic Opti Nerve Sheat Diameter Estimation} % abbreviated title (for running head)
%                                     also used for the TOC unless
%                                     \toctitle is used
%
\author{}

%
\authorrunning{} % abbreviated author list (for running head)
%
%%%% list of authors for the TOC (use if author list has to be modified)
\tocauthor{}
%
\institute{Kitware Inc, Carrboro NC 27510, USA,\\
\email{}
}

\maketitle              % typeset the title of the contribution

\begin{abstract}
\end{abstract}
%
\section{Introduction}
Portable ultrasound techonlogy is uniquley well suited to develop easy-to-use
system that enables emergency responders to quickly assess the severity of a
patient's injuries at accidents and in remote environments in which expert
medical imaging personnel and advanced imaging equipment are not readily
available.

This paper considers the application of point-of-care, computer-assisted
ultrasound system is for in-field traumatic brain injury (TBI) assessment via
the detection of increased intracranial pressure. Delayed treatment of
increased intracranial pressure can cause temporary or permanent brain damage
or even long-term coma and death.  For example, it has been shown that acute
subdural hematomas in severe TBI patients cause significant increase in
intracranial pressure, are associated with 90\% mortality if detected and
treated more than 4 hours after injury, and yet are associated with only 30\%
mortality if detected and treated earlier.

Our system detects elevated intracranial pressure associated with TBI by
automatically measuring the diameter of the optic nerve sheath, a known
indicator of elevated intracranial pressure.  

\section{Related Work}


\section{Algorithm}
In a clinical setting the optic nerve sheath diameter is measured by ocular
ultrasound (see Image). An ultrasound image is acquired and the physicians
manually measures the diameter of optic nerve sheath at a location 3mm behind
the retina. Acquiring such images and making these measurements is challenging
and time consuming task.

Our algorithm reports estimates of the optic nerve sheath diameter in near real
time as an ultrasound probe is swept across the closed eye of a patient. At a
high level, the algorithm proceeds by locating the orb of the eye through
registration of an ellipse with the largest dark circle in the image data. From
the ellipse an approximate location of the optic nerve is constructed and used
to fit two bars to the walls of the optic nerve. This high-level description
leaves out several intermediate steps such as those involving image smoothing,
morphological operations, and distance transforms, that are required to achieve
good registration results. Below are images of results of the fitting procedure
which illustrate that the algorithm runs on a wide variety of images with
differing quality.


This algorithm is now integrated into a user-friendly interface and performs
estimates at 4 frames per second. Depending on the ultrasound probe, the
algorithm can be simplified by skipping the eye estimation step to run at
around 20 frames per second.

\section{Evaluation}
To evaluate the quality of the estimates we had 13 volunteers ranging from
Kitware engineers to medical professionals annotate 23 ultrasound images.  The
image below shows a linear fit of the automatic estimated diameter from two
different parameter settings of the algorithm as compared to a medical
professional. The automatic estimates had an R2 of around 0.9 and 0.8,
respectively. The mean intra correlations among medical professionals was 0.84.
Thus, on this small sample size the automatic estimates perform on par with
medical experts. The correlation among experts matches results of previous
studies.

For additional evaluation of the performance of the algorithm we built a
phantom from gelatine in which 3D-printed “optic nerves” (plastic discs) of
known diameter can be embedded under gelatine orbs and imaged, producing
ultrasound images that look surprising like clinical ocular ultrasound images.

The goal of this evaluation is to check if an accurate estimate of the optic
nerve is possible with the proposed algorithm. In this controlled setting we
first located the optic nerve using the ultrasound. Once a good image of the
optic nerve was achieved the automatic estimation process was started and run
interactively for about 10 seconds which resulted in approximately 40 to 50
optic nerve sheath diameter estimates. The table below shows that the means of
the automatic estimates are within less than +/- 0.5mm  with a relatively tight
distribution of estimates around the ground truth diameter. A study on
intra-operator variations on the same type of phantom, reports standard
deviations of the measurements ranging from 0.25 to 0.6 mm depending on
diameter size.  

\section{Conclusion}
The results presented indicate that the algorithm performs well in a controlled
setting.  A retrospective analysis of clinical images indicate that our system
performs similar to an expert, and a phantom study using 3D printed optic
nerves of known diameter suggests that our system is accurate.

The next step is to integrate an image quality estimation algorithm, so that
high quality images of the optic nerve can be automatically identified as a
probe is swept over the eye. This will eliminate the need for the operator to
view, interpret, or make measurements on an ultrasound image when using it to
assess a dilated optic nerve sheath, as indicative of increased intracranial
pressure. Ultimately, we aim for a system that can be used by novice operators
with minimal ultrasound experience. The system will report when a sufficient
image has been acquired, and it will automatically determine if the diameter of
the optic nerve is greater than 5mm, thereby indicating increased intracranial
pressure associated with mild or more severe TBI [4].

The long term goal of Kitware is to develop a lightweight portable ultrasound
system that, in addition to automated diagnosis of TBI, includes automatic
diagnosis tools for pneumothorax (detached lung) and internal bleeding.

%
% ---- Bibliography ----
%
\begin{thebibliography}{}
%
\bibitem[Du2012]{Du2012}
Dubost, Clément, et al. 
Optic Nerve Sheath Diameter Used as Ultrasonographic Assessment of the Incidence of Raised Intracranial Pressure in PreeclampsiaA Pilot Study.
The Journal of the American Society of Anesthesiologists 116.5 (2012): 1066-1071.

\bibitem[Jo2016]{Jo2016}
Johnson, Garrett GRJ, et al. 
Estimating the accuracy of optic nerve sheath diameter measurement using a pocket-sized, handheld ultrasound on a simulation model.
Critical Ultrasound Journal 8.1 (2016): 18.

\bibitem[Ze2014]{Ze2014}
Zeiler, F. A., et al. 
“A unique model for ONSD Part II: inter/intra-operator variability.” 
Can J Neurol Sci 41 (2014): 430-435.

\bibitem[Ma2015]{Ma2015}
Maissan IM , Dirven PJ , Haitsma IK , Hoeks SE , Gommers D , Stolker RJ 
Ultrasonographic measured optic nerve sheath diameter as an accurate and quick monitor for changes in intracranial pressure. 
J Neurosurg. 2015 Sep;123(3):7437
\end{thebibliography}

\end{document}
